\documentclass[11pt]{article}
\usepackage{listings}

%Gummi|065|=)
\title{\textbf{LEVBDIM: A Light EVent builder based on DIM}}
\author{Laurent Mirabito\\
		}

\date{\today}
\begin{document}

\maketitle
\begin{abstract}
 We developped a light data acquisition system, based on DIM \cite{dim} and mongoose-cpp \cite{mongoose-cpp} frameworks. Providing binary data collection, events building, web accessible finite state machine and process control, it is well suit to manage distribute data source of laboratory or beam test .
\end{abstract}
\section{Introduction}

DIM\cite{dim} is an HEP acquisition framework developped in the DELPHI experiement. It embeds binary packet exchange in messages published by server and subscribed by client. It is light and TCP/IP based anc can be installed on nearly all currently used OS nowdays.

LEVBDIM is a simple acquisition framework where DIM is used to exchange structured binary buffer. It provides several functionnalities oftenly used in modern data acquisition:

\begin{itemize}
    \item Predefine structure of buffer
    \item publication of data source
    \item collection of data source
    \item Event building
    \item Finite State Machine
    \item Web access
    \item process control
\end{itemize}
All those functionnalities are mainly independant and can be used on their own. The last section of this documenation details a full example using all the capabilities of LEVBDIM.

\section{Installation}
\subsection{Packages}
Few standard packages are needed and can be found on any linux distributions: boost, jsoncpp, curl, git and scons

\begin{verbatim}
sudo apt-get -y install libboost-dev libboost-system-dev 
libboost-filesystem-dev libboost-thread-dev libjsoncpp-dev 
libcurl4-gnutls-dev git scons

\end{verbatim}
Two network libraries should be installed. The first one is DIM that can be download and compile from https://dim.web.cern.ch/dim/. The second one is used to give web access to the application, it's Mongoose-cpp\cite{mongoose-cpp} based on Mongoose \cite{mongoose}. One version is distributed with levbdim and can be compile before the installation.

\subsection{LEVBDIM installation}
\subsubsection{Getting the software}
The software is distributed on github\cite{github}. The master version is download with
 \begin{verbatim}
  git clone http://github.com/mirabitl/levbdim.git
\end{verbatim}
\subsubsection{Install the software}
\paragraph{Install Mongoose-cpp}
\begin{verbatim}
 cd levbdim \newline
 source web/mongoose.install
\end{verbatim}


\paragraph{Compile LEVBDIM}

\begin{verbatim}
  cd levbdim
  scons 
\end{verbatim}

\section{Library functionnalities}
\subsection{The Event builder}
Event building consists of merging various data source that collect event fragment at the same time. Each data source should consequently have a localisation tag and a time tag for each data fragment it provides. This fragment are published by a DimService and is centrally collected to build an event, i.e a collection of data fragment with an identical time tag.

The event builder proposed is not itself distributed, it is unique and so no geographical or time dependant partitionning is available. All data sources are collected by a unique process. It subscribes to all available data source and writes them on reciept in a shared memory. A separate thread scans the memory and build the event, i.e a collection of data buffer from each registered data source with the same time tag. A last thread process completed event and call registered \emph{event processors} that can write event to disk in user defined format. The next sections detail the software tools provided to achieve those tasks.
\subsection{The buffer structure}
The \emph{levbdim::buffer} class is a simple data structure that provides space to store the read data (the \emph{payload}) and four tags:
\begin{itemize}
\item The \emph{Detector ID}: A four bytes id used to tag different geographical partition (ex. tracker barrel, ECAL end cap...)

\item The \emph{source ID}: A four bytes id characterizing the data source (ex. ADC or TDC number)
\item The \emph{event number}: A four bytes number of readouts
 
 \item The \emph{bunch crossing number}: an eight bytes number chracterizing the time of the event in clocks count. The clock is the typical clock of the experiment. It may allows a finer events building if needed. It si currently optionnal
 
 \item The \emph{payload}: a bytes array containing detector data for the given \emph{event number} readout. 
 
 \item The \emph{payload size}: the size of the \emph{payload} array.
\end{itemize}

The payload can be compressed on the fly using the gzip\cite{gzip} library.

Buffer do not need to be allocated by data producers since they are embedded in the datasource class.
\subsection{The datasource class}
A \emph{levbdim::datasource} object instantiates a {levbdim::buffer} and creates the associated DimService with the name {\bf /FSM/LEVBDIM/DS-X-Y/DATA}  where X is the detector id and Y the data source one. The buffer has method to be accessed
\begin{thebibliography}{9}

\bibitem{dim}
  C Gaspar,
  \emph{DIM},
  
\bibitem{mongoose-cpp}
\emph{https://github.com/Gregwar/mongoose-cpp}
\bibitem{mongoose}
\emph{https://github.com/cesanta/mongoose}
\bibitem{gzip}
\emph{https://github.com/cesanta/mongoose}
\end{thebibliography}
\end{document}
